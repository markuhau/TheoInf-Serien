%BEGIN_FOLD
%%%%%%%%%%%%%%%%%%%%%%%%%%%%%%%%%%%%%%%%
% Programming/Coding Assignment
% LaTeX Template
%
% Original author:
% Ted Pavlic (http://www.tedpavlic.com)
%
%
% This template uses a Perl script as an example snippet of code, most other
% languages are also usable. Configure them in the "CODE INCLUSION 
% CONFIGURATION" section.
%
%%%%%%%%%%%%%%%%%%%%%%%%%%%%%%%%%%%%%%%%%
%END_FOLD
%----------------------------------------------------------------------------------------
%BEGIN_FOLD	%PACKAGES AND OTHER DOCUMENT CONFIGURATIONS
%----------------------------------------------------------------------------------------

\documentclass{article}
\usepackage[utf8]{inputenc}

\usepackage[ngerman]{babel}
\usepackage{amsmath}
\usepackage{amssymb}
\usepackage{mathtools}
\usepackage{amsfonts}
\usepackage{fancyhdr} % Required for custom headers
\usepackage{lastpage} % Required to determine the last page for the footer
\usepackage{extramarks} % Required for headers and footers
\usepackage[usenames,dvipsnames]{color} % Required for custom colors
\usepackage{graphicx} % Required to insert images
\usepackage{listings} % Required for insertion of code
\usepackage{courier} % Required for the courier font
\usepackage{enumerate} % used for enumerate args
\usepackage{multicol} % columns

\usepackage{pgf} 
\usepackage{tikz}
\usepackage{forest} % treees :D
\usetikzlibrary{arrows,automata} %for FSM
\usetikzlibrary{positioning}
\usepackage{mathtools}

% Custom commands
\DeclareMathOperator{\Kl}{Kl} %Klassen von Zuständen
\newcommand{\Ebool}{\{0, 1\}}
\DeclarePairedDelimiter{\set}{\{}{\}}
\DeclarePairedDelimiter{\ceil}{\lceil}{\rceil}
% Shamelessly copied from http://tex.stackexchange.com/questions/43008/absolute-value-symbols
\DeclarePairedDelimiter\abs{\lvert}{\rvert} % nice |x|
\DeclarePairedDelimiter\norm{\lVert}{\rVert} % nice ||x||
% Swap the definition of \abs* and \norm*, so that \abs
% and \norm resizes the size of the brackets, and the 
% starred version does not.
%END_FOLD
%----------------------------------------------------------------------------------------
%BEGIN_FOLD
\makeatletter
\let\oldabs\abs
\def\abs{\@ifstar{\oldabs}{\oldabs*}}
\let\oldnorm\norm
\def\norm{\@ifstar{\oldnorm}{\oldnorm*}}
\makeatother


% Margins
\topmargin=-0.45in
\evensidemargin=0in
\oddsidemargin=0in
\textwidth=6.5in
\textheight=9.0in
\headsep=0.25in

\linespread{1.1} % Line spacing

% Set up the header and footer
\pagestyle{fancy}
\lhead{\hmwkAuthorName} % Top left header
%\chead{\hmwkClass\ (\hmwkClassInstructor\): \hmwkTitle} % Top center head
%\rhead{\firstxmark} % Top right header
\rhead{}
\lfoot{\lastxmark} % Bottom left footer
\cfoot{} % Bottom center footer
\rfoot{Seite\ \thepage\ von\ \protect\pageref{LastPage}} % Bottom right footer
\renewcommand\headrulewidth{0.4pt} % Size of the header rule
\renewcommand\footrulewidth{0.4pt} % Size of the footer rule

\setlength\parindent{0pt} % Removes all indentation from paragraphs

%----------------------------------------------------------------------------------------
%	CODE INCLUSION CONFIGURATION
%----------------------------------------------------------------------------------------

\definecolor{MyDarkGreen}{rgb}{0.0,0.4,0.0} % This is the color used for comments
\lstloadlanguages{Pascal} % Load Pascal syntax for listings, for a list of other languages supported see: ftp://ftp.tex.ac.uk/tex-archive/macros/latex/contrib/listings/listings.pdf
\lstset{language=Perl, % Use Pascal in this example
	frame=single, % Single frame around code
	basicstyle=\small\ttfamily, % Use small true type font
	keywordstyle=[1]\color{Blue}\bf, % Pascal functions bold and blue
	keywordstyle=[2]\color{Purple}, % Pascal function arguments purple
	keywordstyle=[3]\color{Blue}\underbar, % Custom functions underlined and blue
	identifierstyle=, % Nothing special about identifiers                                         
	commentstyle=\usefont{T1}{pcr}{m}{sl}\color{MyDarkGreen}\small, % Comments small dark green courier font
	stringstyle=\color{Purple}, % Strings are purple
	showstringspaces=false, % Don't put marks in string spaces
	tabsize=5, % 5 spaces per tab
	%
	% Put standard Pascal functions not included in the default language here
	morekeywords={rand},
	%
	% Put Pascal function parameters here
	morekeywords=[2]{on, off, interp},
	%
	% Put user defined functions here
	morekeywords=[3]{test},
	%
	morecomment=[l][\color{Blue}]{...}, % Line continuation (...) like blue comment
	numbers=left, % Line numbers on left
	firstnumber=1, % Line numbers start with line 1
	numberstyle=\tiny\color{Blue}, % Line numbers are blue and small
	stepnumber=5 % Line numbers go in steps of 5
}

% Creates a new command to include a perl script, the first parameter is the filename of the script (without .p), the second parameter is the caption
\newcommand{\pascalscript}[2]{
	\begin{itemize}
		\item[]\lstinputlisting[caption=#2,label=#1]{#1.p}
	\end{itemize}
}

%----------------------------------------------------------------------------------------
%	DOCUMENT STRUCTURE COMMANDS
%	Skip this unless you know what you're doing
%----------------------------------------------------------------------------------------

% Header and footer for when a page split occurs within a problem environment
%\newcommand{\enterProblemHeader}[1]{
%\nobreak\extramarks{#1}{#1 continued on next page\ldots}\nobreak
%\nobreak\extramarks{#1 (continued)}{#1 continued on next page\ldots}\nobreak
%}

% Header and footer for when a page split occurs between problem environments
%\newcommand{\exitProblemHeader}[1]{
%\nobreak\extramarks{#1 (continued)}{#1 continued on next page\ldots}\nobreak
%\nobreak\extramarks{#1}{}\nobreak
%}

\setcounter{secnumdepth}{0} % Removes default section numbers
\newcounter{homeworkProblemCounter} % Creates a counter to keep track of the number of problems

\newcommand{\homeworkProblemName}{}
%END_FOLD
\newenvironment{homeworkProblem}[1][Aufgabe \arabic{homeworkProblemCounter}]
%BEGIN_FOLD
{ % Makes a new environment called homeworkProblem which takes 1 argument (custom name) but the default is "Problem #"
	
	\stepcounter{homeworkProblemCounter} % Increase counter for number of problems
	\renewcommand{\homeworkProblemName}{#1} % Assign \homeworkProblemName the name of the problem
	\section{\homeworkProblemName} % Make a section in the document with the custom problem count
	%\enterProblemHeader{\homeworkProblemName} % Header and footer within the environment
}{
%\exitProblemHeader{\homeworkProblemName} % Header and footer after the environment
}

\newcommand{\problemAnswer}[1]{ % Defines the problem answer command with the content as the only argument
	\noindent\framebox[\columnwidth][c]{\begin{minipage}{0.98\columnwidth}#1\end{minipage}} % Makes the box around the problem answer and puts the content inside
}

\newcommand{\homeworkSectionName}{}
\newenvironment{homeworkSection}[1]{ % New environment for sections within homework problems, takes 1 argument - the name of the section
	\renewcommand{\homeworkSectionName}{#1} % Assign \homeworkSectionName to the name of the section from the environment argument
	\subsection{\homeworkSectionName} % Make a subsection with the custom name of the subsection
	%\enterProblemHeader{\homeworkProblemName\ [\homeworkSectionName]} % Header and footer within the environment
}{
%\enterProblemHeader{\homeworkProblemName} % Header and footer after the environment
}
%END_FOLD
%----------------------------------------------------------------------------------------
%BEGIN_FOLD %NAME AND CLASS SECTION
%----------------------------------------------------------------------------------------

\newcommand{\hmwkTitle}{Blatt} % Assignment title
\newcommand{\hmwkDueDate}{16.\ Oktober\ 2015} % Due date
\newcommand{\hmwkClass}{Theoretische Informatik} % Course/class
\newcommand{\hmwkClassInstructor}{} % Teacher/lecturer
\newcommand{\hmwkAuthorName}{Linus Fessler, Markus Hauptner, Philipp Schimmelfennig} % Your name
\newcommand{\hmwkNumber}{4}
\newcommand{\hmwkAssiInfo}{Assistent: Sascha Krug, CHN D 42}
%END_FOLD
%----------------------------------------------------------------------------------------
%BEGIN_FOLD
%----------------------------------------------------------------------------------------
%	TITLE PAGE
%----------------------------------------------------------------------------------------

\title{
	\vspace{2in}
	\textmd{\textbf{\hmwkClass:\ \hmwkTitle\ \hmwkNumber}}\\
	\normalsize\vspace{0.1in}\small{Abgabe\ bis\ \hmwkDueDate}
	\\\hmwkAssiInfo
	\\
	\vspace{0.1in}\large{\textit{\hmwkClassInstructor}
		\vspace{3in}
	}}
	\author{\textbf{\hmwkAuthorName}}
	\date{} % Insert date here if you want it to appear below your name
	
	%----------------------------------------------------------------------------------------
	%END_FOLD
	\begin{document}
		%BEGIN_FOLD
		\maketitle
		
		%----------------------------------------------------------------------------------------
		%	TABLE OF CONTENTS
		%----------------------------------------------------------------------------------------
		
		%\setcounter{tocdepth}{1} % Uncomment this line if you don't want subsections listed in the ToC
		%END_FOLD
		%----------------------------------------------------------------------------------------
		\addtocounter{homeworkProblemCounter}{9}
		%----------------------------------------------------------------------------------------
		\newpage
		%\tableofcontents
		%\newpage
		
		%----------------------------------------------------------------------------------------
		%	Aufgabe S10
		%----------------------------------------------------------------------------------------
		
		\begin{homeworkProblem}
			\begin{enumerate}[(a)]
				\item
				$L=\set{0^{\binom{2n}{n}}\mid n\in \mathbb{N}}$\\
				Wir machen einen Widerspruchsbeweis. \textit{Annahme:} $L$ ist regulär.\\
				Wir betrachten die Wörter 
				\[
				0^{\binom{2\cdot1}{1}+1},\, 0^{\binom{2\cdot2}{2}+1},\, \cdots
				\]
				Das erste Wort in $L_{0^{\binom{2m}{m}+1}} = \set{y\mid 0^{\binom{2m}{m}+1}y \in L}$ ist $0^{\binom{2(m+1)}{m+1}-\left(\binom{2m}{m}+1\right)}$\\
				Nach \textit{Satz 3.1} ist die Kolmogorov-Komplexität $K(y)\leq 1+c$.\\
				Es gibt unendlich viele Wörter $0^{\binom{2(m+1)}{m+1}-\left(\binom{2m}{m}+1\right)}$, da der Term mit $m$ wächst,
				aber nur endlich viele Programme mit Länge $\leq 1+c$.\\
				Also haben wir einen Widerspruch $\Rightarrow$ die Annahme war falsch $\Rightarrow$ L ist nicht regulär.		
			
				\item
				Wir machen einen Widerspruchsbeweis. \textit{Annahme:} $L$ ist regulär.$\;\Rightarrow\;$ Es gibt einen EA $A=(Q, \Sigma, \delta_A, q_o, F)$ mit $L(A)=L$. Sei $m=|Q|$\\
				Betrachten wir die Wörter
				\[
				\lambda, b, b^2, \cdots, b^m
				\] 
				Das sind mehr Wörter, als $A$ Zustände hat.
				$\Rightarrow \exists i, j\;\; j\not=i$, sodass $\hat{d}(q_0, b^i)=\hat{d}(q_0, b^j)$\\
				Also gilt nach \textit{Lemma 3.3}
				\[ b^iz \in L  \leftrightarrow b^jz \in L \quad \forall z \in \Ebool^*
				\]
				Sei $z=a^{2i}$, dann gilt $b^iz=b^ia^{2i} \in L$ aber für $i\not=j$,\; $b^ja^{2i} \not\in L$\\
				Also haben wir einen Widerspruch $\Rightarrow$ die Annahme war falsch $\Rightarrow$ L ist nicht regulär.
			\end{enumerate}
		\end{homeworkProblem}
		%------------------Aufgabe 11---------------------------------
		\begin{homeworkProblem}
			\begin{enumerate}[(a)]
				\item
				Wir machen einen Widerspruchsbeweis. \textit{Annahme:} $L$ ist regulär. Dann gilt das \textit{Pumping-Lemma} für $L$.\\
				Wir betrachten nun das Wort
				\[
				w=0^{n_0}1^{n_0}
				\]
				Offensichtlich gilt $|w| \leq n_0$.\\
				Daher gilt für die Zerlegung $w=yxz$ nach $(i)$ und $(ii)$, dass $y=0^l$, $x=0^m$, $l+m \leq n_0$.\\
				Weil $w=yxz=0^{n_0}1^{n_0} \not\in L$ müssen nach $(iii)$ auch alle $w\in\{yx^kz \mid k\in \mathbb{N}\} \not\in L$ sein. 
				Wenn wir nun das Wort
				\[
				w=yx^2z=0^l0^{2m}z
				\]
				betrachten, dann hat sich die Anzahl der Nullen erhöht, die Anzahl der Einsen ist jedoch gleich geblieben. Dadurch ist jedoch nach Definition $w\in L$.\\
				Es gibt ein Widerspruch $\Rightarrow$ Die Annahme war falsch $\Rightarrow$ $L$ ist nicht regulär.
			\end{enumerate}	
		\end{homeworkProblem}
		
		\begin{homeworkProblem}
			\begin{enumerate}[(a)]
				\item
				
				Ein NEA für die Sprache $L = \{x \in \{0, 1\}^* \mid |x|_1 \text{ mod } 3 = 0 \text{ oder } x \text{ enthält ein Teilwort } 1y1 \text{ für } y \in \{0, 1\}^2\}$ sieht so aus:
				
				\begin{center}
					\begin{tikzpicture}[->,>=stealth',shorten >=1pt,auto,node distance=2.8cm,
					semithick,every node/.style={scale=1}]
					\tikzstyle{every state}=[circle,text=black]
					
					\node[initial,state,accepting]	(1)             		{$\lambda$};
					\node[state,accepting]			(2) [above right of=1]	{$q_0$};
					\node[state]					(3) [right of=2]		{$q_1$};
					\node[state]					(4) [right of=3]		{$q_2$};
					\node[state]					(5) [below right of=1]	{$p_0$};
					\node[state]					(6) [right of=5]		{$p_1$};
					\node[state]					(7) [right of=6]		{$p_2$};
					\node[state]					(8) [right of=7]		{$p_3$};
					\node[state,accepting]			(9) [right of=8]		{$p_4$};
					
					\path	(1)		edge  								node {0}	(2)
					(1)		edge 								node {1}	(3)
					(1)		edge [below left]					node {0,1}	(5)
					(1)		edge  								node {1}	(6)
					(2)		edge 								node {1}	(3)
					(2)		edge [loop above]					node {0}	(2)
					(3)		edge 								node {1}	(4)
					(3)		edge [loop above]					node {0}	(3)
					(4)		edge [bend left]					node {1}	(2)
					(4)		edge [loop above]					node {0}	(4)
					(5)		edge [loop below]					node {0,1}	(5)
					(5)		edge								node {1}	(6)
					(6)		edge 								node {0,1}	(7)
					(7)		edge 								node {0,1}	(8)
					(8)		edge 								node {1}	(9)
					(9)		edge [loop below]					node {0,1}	(9)
					;
					\end{tikzpicture}
				\end{center}
				
				Die Idee des Entwurfs ist, dass der NEA aus zwei Teil-Automaten besteht, die je folgende Bedingungen prüfen:
				\begin{equation} |x|_1 \text{ mod } 3 = 0 \end{equation}
				\begin{equation} x \text{ enthält ein Teilwort } 1y1 \text{ für } y \in \{0, 1\}^2\} \end{equation}
				Der Automat, der Bedingung (1) überprüft, ist durch die Zustände $\{q_0,q_1,q_2\}$ gegeben und derjenge, der (2) überprüft, durch die Zustände $\{p_0,p_1,p_2,p_3,p_4\}$.
				
				Im Startzustand $\lambda$ kann sich der NEA nun entscheiden, ob er Bedingung (1) oder (2) überprüft.
				
				\item
				Die Übertragungsfunktion $\delta'$ des äquivalenten deterministischen Automaten lautet wie folgt:
				\begin{center}
					\begin{tabular}{l|l|l}
						$\delta'$ & $a$ & $b$ \\
						\hline
						$p'=\{p\}$ & $\{p,q\}$ & $\{p\}$ \\
						$q'=\{p,q\}$ & $\{p,q,r\}$ & $\{p,r\}$ \\
						$r'=\{p,r\}$ & $\{p,q,s\}$ & $\{p\}$ \\
						$s'=\{p,s\}$ & $\{p,s\}$ & $\{p,r\}$ \\
						 $t'=\{p,q,r\}$ & $\{p,r,s\}$ & $\{p,r\}$ \\
						 $v'=\{p,q,s\}$ & $\{p,r,s\}$ & $\{p,r,s\}$ \\
						 $u'=\{p,r,s\}$ & $\{p,s\}$ & $\{p,s\}$
					\end{tabular}
				\end{center}
				Daraus folgt der deterministische Automat: 
				\begin{center}
					\begin{tikzpicture}[->,>=stealth',shorten >=1pt,auto,node distance=2.8cm,
					semithick,every node/.style={scale=1}]
					\tikzstyle{every state}=[circle,text=black]
					
					\node[initial,state] 			(1)             	{$p'$};
					\node[state]					(2) [right of=1]	{$q'$};
					\node[state]					(3) [right of=2]	{$r'$};
					\node[state,accepting]			(4) [right of=3]	{$s'$};
					\node[state]					(5) [below of=2]	{$t'$};
					\node[state]					(6) [below of=3]	{$u'$};
					\node[state]					(7) [below of=4]	{$v'$};
					
					\path	(1)		edge 						node {a}	(2)
							edge [loop above]			node {b}	(1)
					(2)		edge 						node {a}	(5)
							edge 						node {b}	(3)
					(3)		edge 						node {a}	(6)
							edge [bend right,above]		node {b}	(1)
					(4)		edge [loop above]			node {a,b}	(4)
					(5)		edge 						node {b}	(3)
							edge [bend right,below]		node {a}	(7)
					(6)		edge 						node {a,b}	(7)
					(7)		edge 						node {a,b}	(4)
					;
					\end{tikzpicture}
				\end{center}
			\end{enumerate}
		\end{homeworkProblem}
	\end{document}