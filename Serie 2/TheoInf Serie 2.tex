%%%%%%%%%%%%%%%%%%%%%%%%%%%%%%%%%%%%%%%%
% Programming/Coding Assignment
% LaTeX Template
%
% Original author:
% Ted Pavlic (http://www.tedpavlic.com)
%
%
% This template uses a Perl script as an example snippet of code, most other
% languages are also usable. Configure them in the "CODE INCLUSION 
% CONFIGURATION" section.
%
%%%%%%%%%%%%%%%%%%%%%%%%%%%%%%%%%%%%%%%%%

%----------------------------------------------------------------------------------------
%	PACKAGES AND OTHER DOCUMENT CONFIGURATIONS
%----------------------------------------------------------------------------------------

\documentclass{article}
\usepackage[utf8]{inputenc}

\usepackage[english]{babel}
\usepackage{amsmath}
\usepackage{amssymb}
\usepackage{mathtools}
\usepackage{amsfonts}
\usepackage{fancyhdr} % Required for custom headers
\usepackage{extramarks} % Required for headers and footers
\usepackage[usenames,dvipsnames]{color} % Required for custom colors
\usepackage{graphicx} % Required to insert images
\usepackage{listings} % Required for insertion of code
\usepackage{courier} % Required for the courier font
\usepackage{enumerate} % used for enumerate args
\usepackage{multicol} % columns

\usepackage{xfrac}
\usepackage{pgf} 
\usepackage{tikz}
\usetikzlibrary{arrows,automata} %for FSM

% Custom commands
\DeclareMathOperator{\Kl}{Kl} %Klassen von Zuständen

\usepackage{mathtools}
\DeclarePairedDelimiter{\ceil}{\lceil}{\rceil}
% Shamelessly copied from http://tex.stackexchange.com/questions/43008/absolute-value-symbols
\DeclarePairedDelimiter\abs{\lvert}{\rvert} % nice |x|
\DeclarePairedDelimiter\norm{\lVert}{\rVert} % nice ||x||
% Swap the definition of \abs* and \norm*, so that \abs
% and \norm resizes the size of the brackets, and the 
% starred version does not.
\makeatletter
\let\oldabs\abs
\def\abs{\@ifstar{\oldabs}{\oldabs*}}
\let\oldnorm\norm
\def\norm{\@ifstar{\oldnorm}{\oldnorm*}}
\makeatother


% Margins
\topmargin=-0.45in
\evensidemargin=0in
\oddsidemargin=0in
\textwidth=6.5in
\textheight=9.0in
\headsep=0.25in

\linespread{1.1} % Line spacing

% Set up the header and footer
\pagestyle{fancy}
\lhead{\hmwkAuthorName} % Top left header
%\chead{\hmwkClass\ (\hmwkClassInstructor\): \hmwkTitle} % Top center head
%\rhead{\firstxmark} % Top right header
\rhead{}
\lfoot{\lastxmark} % Bottom left footer
\cfoot{} % Bottom center footer
\rfoot{Seite\ \thepage\ von\ \protect\pageref{LastPage}} % Bottom right footer
\renewcommand\headrulewidth{0.4pt} % Size of the header rule
\renewcommand\footrulewidth{0.4pt} % Size of the footer rule

\setlength\parindent{0pt} % Removes all indentation from paragraphs
%\DeclarePairedDelimiter\ceil{\lceil}{\rceil}
\DeclarePairedDelimiter\floor{\lfloor}{\rfloor}

%----------------------------------------------------------------------------------------
%	CODE INCLUSION CONFIGURATION
%----------------------------------------------------------------------------------------

\definecolor{MyDarkGreen}{rgb}{0.0,0.4,0.0} % This is the color used for comments
\lstloadlanguages{Pascal} % Load Pascal syntax for listings, for a list of other languages supported see: ftp://ftp.tex.ac.uk/tex-archive/macros/latex/contrib/listings/listings.pdf
\lstset{language=Perl, % Use Pascal in this example
        frame=single, % Single frame around code
        basicstyle=\small\ttfamily, % Use small true type font
        keywordstyle=[1]\color{Blue}\bf, % Pascal functions bold and blue
        keywordstyle=[2]\color{Purple}, % Pascal function arguments purple
        keywordstyle=[3]\color{Blue}\underbar, % Custom functions underlined and blue
        identifierstyle=, % Nothing special about identifiers                                         
        commentstyle=\usefont{T1}{pcr}{m}{sl}\color{MyDarkGreen}\small, % Comments small dark green courier font
        stringstyle=\color{Purple}, % Strings are purple
        showstringspaces=false, % Don't put marks in string spaces
        tabsize=5, % 5 spaces per tab
        %
        % Put standard Pascal functions not included in the default language here
        morekeywords={rand},
        %
        % Put Pascal function parameters here
        morekeywords=[2]{on, off, interp},
        %
        % Put user defined functions here
        morekeywords=[3]{test},
        %
        morecomment=[l][\color{Blue}]{...}, % Line continuation (...) like blue comment
        numbers=left, % Line numbers on left
        firstnumber=1, % Line numbers start with line 1
        numberstyle=\tiny\color{Blue}, % Line numbers are blue and small
        stepnumber=5 % Line numbers go in steps of 5
}

% Creates a new command to include a perl script, the first parameter is the filename of the script (without .p), the second parameter is the caption
\newcommand{\pascalscript}[2]{
\begin{itemize}
\item[]\lstinputlisting[caption=#2,label=#1]{#1.p}
\end{itemize}
}

%----------------------------------------------------------------------------------------
%	DOCUMENT STRUCTURE COMMANDS
%	Skip this unless you know what you're doing
%----------------------------------------------------------------------------------------

% Header and footer for when a page split occurs within a problem environment
%\newcommand{\enterProblemHeader}[1]{
%\nobreak\extramarks{#1}{#1 continued on next page\ldots}\nobreak
%\nobreak\extramarks{#1 (continued)}{#1 continued on next page\ldots}\nobreak
%}

% Header and footer for when a page split occurs between problem environments
%\newcommand{\exitProblemHeader}[1]{
%\nobreak\extramarks{#1 (continued)}{#1 continued on next page\ldots}\nobreak
%\nobreak\extramarks{#1}{}\nobreak
%}

\setcounter{secnumdepth}{0} % Removes default section numbers
\newcounter{homeworkProblemCounter} % Creates a counter to keep track of the number of problems

\newcommand{\homeworkProblemName}{}
\newenvironment{homeworkProblem}[1][Aufgabe \arabic{homeworkProblemCounter}]{ % Makes a new environment called homeworkProblem which takes 1 argument (custom name) but the default is "Problem #"
\stepcounter{homeworkProblemCounter} % Increase counter for number of problems
\renewcommand{\homeworkProblemName}{#1} % Assign \homeworkProblemName the name of the problem
\section{\homeworkProblemName} % Make a section in the document with the custom problem count
%\enterProblemHeader{\homeworkProblemName} % Header and footer within the environment
}{
%\exitProblemHeader{\homeworkProblemName} % Header and footer after the environment
}

\newcommand{\problemAnswer}[1]{ % Defines the problem answer command with the content as the only argument
\noindent\framebox[\columnwidth][c]{\begin{minipage}{0.98\columnwidth}#1\end{minipage}} % Makes the box around the problem answer and puts the content inside
}

\newcommand{\homeworkSectionName}{}
\newenvironment{homeworkSection}[1]{ % New environment for sections within homework problems, takes 1 argument - the name of the section
\renewcommand{\homeworkSectionName}{#1} % Assign \homeworkSectionName to the name of the section from the environment argument
\subsection{\homeworkSectionName} % Make a subsection with the custom name of the subsection
%\enterProblemHeader{\homeworkProblemName\ [\homeworkSectionName]} % Header and footer within the environment
}{
%\enterProblemHeader{\homeworkProblemName} % Header and footer after the environment
}

%----------------------------------------------------------------------------------------
%	NAME AND CLASS SECTION
%----------------------------------------------------------------------------------------

\newcommand{\hmwkTitle}{Blatt} % Assignment title
\newcommand{\hmwkDueDate}{2.\ Oktober\ 2015} % Due date
\newcommand{\hmwkClass}{Theoretische Informatik} % Course/class
\newcommand{\hmwkClassInstructor}{} % Teacher/lecturer
\newcommand{\hmwkAuthorName}{Linus Fessler, Markus Hauptner, Philipp Schimmelfennig} % Your name
\newcommand{\hmwkNumber}{2}

%----------------------------------------------------------------------------------------
%	TITLE PAGE
%----------------------------------------------------------------------------------------

\title{
\vspace{2in}
\textmd{\textbf{\hmwkClass:\ \hmwkTitle\ \hmwkNumber}}\\
\normalsize\vspace{0.1in}\small{Abgabe\ bis\ \hmwkDueDate}
\\Assistent: Sascha Krug, CHN D 42
\\
\vspace{0.1in}\large{\textit{\hmwkClassInstructor}
\vspace{3in}
}}
\author{\textbf{\hmwkAuthorName}}
\date{} % Insert date here if you want it to appear below your name

%----------------------------------------------------------------------------------------

\begin{document}

\maketitle

%----------------------------------------------------------------------------------------
%	TABLE OF CONTENTS
%----------------------------------------------------------------------------------------

%\setcounter{tocdepth}{1} % Uncomment this line if you don't want subsections listed in the ToC

\addtocounter{homeworkProblemCounter}{3}
\newpage
%\tableofcontents
%\newpage

%----------------------------------------------------------------------------------------
%	Aufgabe S4
%----------------------------------------------------------------------------------------

\begin{homeworkProblem}
\begin{enumerate}[(a)]
	\item
	Das folgende Programm $W_n$ generiert das Wort $w_n$:
	
	\begin{lstlisting}[language=pascal,tabsize=2,escapeinside={§}{§}]
§$W_n$§:		begin
				M := n;
				M := 5 §$\times$§ M §$\times$§ M;
				M := 2 ^ (2 ^ M);
				for I = 1 to M do
					write(0);
			end
	\end{lstlisting}
	
	Für $ w_n = 0^{2^{2^{5\cdot n^{2}}}} \in \{0,\,1\}^* $ gilt:
	\begin{alignat*}{2}
		& & |w_n|&=2^{2^{5\cdot n^2}}\\
		&\Rightarrow & log_2\ |w_n|&=2^{5\cdot n^2}\\
		&\Rightarrow & log_2\ log_2\ |w_n|&= 5\cdot n^2\\
		&\Rightarrow & \frac{1}{5}\cdot log_2\ log_2\ |w_n| &=n^2\\
		&\Rightarrow \quad & \sqrt{\frac{1}{5}\cdot log_2\ log_2\ |w_n|}&=n
	\end{alignat*}
	
	Damit ist die Kolmogorov Komplexität:
	\begin{equation*}
	K(z_n)\leq \ceil{log_2(n+1)}+d \leq \ceil{log_2\ n}+d' \leq \lceil \,log_2\left( \sqrt{\frac{1}{5}\cdot log_2\ log_2\ |w_n|\,}\;\right)\rceil+d' = \lceil \,\frac{1}{2} \cdot log_2\left( \frac{1}{5}\cdot log_2\ log_2\ |w_n|\,\;\right)\rceil+d'
	\end{equation*}
	
	\item
	Für die Binärdarstellung einer Zahl der Länge $n$ werden $\lceil log_2(n+1) \rceil$ Bits benötigt. Daher gilt für $K(y_i) \leq \lceil log_2log_2log_2\sqrt{y_i} \rceil + c$:
	\begin{alignat*}{2}
		& & log_2log_2\sqrt{y_i} &= i\\
		& \Rightarrow & log_2\sqrt{y_i} &= 2^i\\
		& \Rightarrow & \sqrt{y_i} &= 2^{2^i}\\
		& \Rightarrow & y_i &= (2^{2^i})^2 = 2^{2^i} \cdot 2^{2^i} = 2^{2^i+2^i} = 2^{2 \cdot 2^i} = 2^{2^{i+1}}
	\end{alignat*}
	
	Das folgende Programm $Y_i$ generiert das Wort $y_i$:
		
	\begin{lstlisting}[language=pascal,tabsize=2,escapeinside={§}{§}]
§$Y_i$§:		begin
				M := i;
				M := 2 ^ (M + 1);
				for K = 1 to M do
					write(10);
			end
	\end{lstlisting}
	Binär dargestellt ist $y_i = 10^{2^{i+1}}$, wobei $y_i < y_{i+1}$ nach kanonischer Ordnung trivialerweise erfüllt ist, da sich die Länge von $y_i$ in jedem Schritt verdoppelt.
\end{enumerate}
\end{homeworkProblem}
\newpage

\begin{homeworkProblem}
	Wir betrachten Wörter über dem Alphabet $\Sigma_{bool}^*=\{0,\,1\}^*$ mit höchstens Länge n: $|w|\leq n$.\\
	Nach der Definiton von Zufälligkeit ist $K(x_n) \geq |w_n| \Leftrightarrow x_n$ ist zufällig.\\
	Also soll $K(x_n) < |w_n|$ sein, damit $w_n$ komprimierbar ist.
	Jedes Programm, das solch ein Wort darstellt, ist selbst eine Bit-Sequenz in $\Sigma_{bool}$ mit Länge bis $n-1$. Nehmen wir an, dass jede Bit-Sequenz in $\{\Sigma_{bool}\}^{n-1}$ ein sinnvolles Programm sei, das ein Wort generieren kann, dann ist die maximale Anzahl dieser Programme
	\begin{equation*}
		\sum_{k=1}^{n-1} 2^k = 2^n-2.
	\end{equation*}
	Die Anzahl aller möglichen Wörter bis Länge $n$ ist:
	\begin{equation*}
		1+\sum_{k=1}^n 2^k = 2^n-2+2^n.
	\end{equation*}
	Falls ein Programm ein Wort komprimiert, gibt es eine Bijektion zwischen diesem Wort und dem Programm.
	Ziehen wir von der Anzahl aller möglicher Wörter die Anzahl aller  möglicher -- diese Wörter komprimierender -- Programme ab, bleiben noch
	\begin{equation*}
		1+\sum_{k=1}^n 2^k \;-\; \sum_{k=1}^{n-1} 2^k= 1+2^n-2+2^n- (2^n-2)=1+2^n
	\end{equation*}
	Wörter übrig, die nicht von Programmen in  $\{\Sigma_{bool}\}^{n-1}$ generiert werden können.
	Dies sind mehr als die $2^n-2$ möglichen Programme, also als die Anzahl potentiell komprimierbarer Wörter.
	Daher ist mehr als die Hälfte der Wörter nicht komprimierbar, also zufällig.
\end{homeworkProblem}
\begin{homeworkProblem}
	Das Wort $w=1^i0^j1^k$ ist eindeutig definiert durch $i$ und $k$, da $i+j=2k \Rightarrow j = 2k-i$. Als Eingabe für unser Programm wählen wir eine Codierung, die $i, k$ beinhaltet:
	\begin{equation*}
	X=\overline{\text{Bin}}(i)\text{Bin}(k)=i_10i_20i_30\cdots a_{\lceil log_2 i+1 \rceil}1k_1k_2k_3\cdots k_{\lceil log_2 k+1 \rceil}
	\end{equation*}
	Da $i \leq 2k$ ist, ist die Eingabelänge beschränkt. Damit ist die Länge der Eingabe
	\begin{alignat*}{2}
		|X| &&=2\left(\lceil\log_2\left(i+1\right)\rceil\right)+\left\lceil\log_2{\left(k+1\right)}\right\rceil
	\end{alignat*}
	und das Programm hat Kolmogorov Komplexität
	\begin{alignat*}{1}
		K(w)&\leq 2\cdot\lceil\log_2\left(i+1\right)\rceil+\left\lceil\log_2{\left(k+1\right)}\right\rceil + c\\
		&\leq 2\cdot\log_2\left(i\right)+\log_2{\left(k\right)} + c'\\
		&\leq 2\cdot\log_2\left(2k\right)+\log_2{\left(k\right)} + c'.\\
	\end{alignat*}
\end{homeworkProblem}

\end{document}
