%%%%%%%%%%%%%%%%%%%%%%%%%%%%%%%%%%%%%%%%
% Programming/Coding Assignment
% LaTeX Template
%
% Original author:
% Ted Pavlic (http://www.tedpavlic.com)
%
%
% This template uses a Perl script as an example snippet of code, most other
% languages are also usable. Configure them in the "CODE INCLUSION 
% CONFIGURATION" section.
%
%%%%%%%%%%%%%%%%%%%%%%%%%%%%%%%%%%%%%%%%%

%----------------------------------------------------------------------------------------
%	PACKAGES AND OTHER DOCUMENT CONFIGURATIONS
%----------------------------------------------------------------------------------------

\documentclass{article}
\usepackage[utf8]{inputenc}

\usepackage[english]{babel}
\usepackage{amsmath}
\usepackage{amssymb}
\usepackage{amsfonts}
\usepackage{fancyhdr} % Required for custom headers
\usepackage{extramarks} % Required for headers and footers
\usepackage[usenames,dvipsnames]{color} % Required for custom colors
\usepackage{graphicx} % Required to insert images
\usepackage{listings} % Required for insertion of code
\usepackage{courier} % Required for the courier font
\usepackage{enumerate} % used for enumerate args
\usepackage{multicol} % columns

\usepackage{pgf} 
\usepackage{tikz}
\usetikzlibrary{arrows,automata} %for FSM
\usepackage{amsmath} 
\usepackage{mathtools}

% Custom commands
\DeclareMathOperator{\Kl}{Kl} %Klassen von Zuständen

\usepackage{mathtools}
\DeclarePairedDelimiter{\ceil}{\lceil}{\rceil}
% Shamelessly copied from http://tex.stackexchange.com/questions/43008/absolute-value-symbols
\DeclarePairedDelimiter\abs{\lvert}{\rvert} % nice |x|
\DeclarePairedDelimiter\norm{\lVert}{\rVert} % nice ||x||
% Swap the definition of \abs* and \norm*, so that \abs
% and \norm resizes the size of the brackets, and the 
% starred version does not.
\makeatletter
\let\oldabs\abs
\def\abs{\@ifstar{\oldabs}{\oldabs*}}
\let\oldnorm\norm
\def\norm{\@ifstar{\oldnorm}{\oldnorm*}}
\makeatother


% Margins
\topmargin=-0.45in
\evensidemargin=0in
\oddsidemargin=0in
\textwidth=6.5in
\textheight=9.0in
\headsep=0.25in

\linespread{1.1} % Line spacing

% Set up the header and footer
\pagestyle{fancy}
\lhead{\hmwkAuthorName} % Top left header
%\chead{\hmwkClass\ (\hmwkClassInstructor\): \hmwkTitle} % Top center head
%\rhead{\firstxmark} % Top right header
\rhead{}
\lfoot{\lastxmark} % Bottom left footer
\cfoot{} % Bottom center footer
\rfoot{Seite\ \thepage\ von\ \protect\pageref{LastPage}} % Bottom right footer
\renewcommand\headrulewidth{0.4pt} % Size of the header rule
\renewcommand\footrulewidth{0.4pt} % Size of the footer rule

\setlength\parindent{0pt} % Removes all indentation from paragraphs

%----------------------------------------------------------------------------------------
%	CODE INCLUSION CONFIGURATION
%----------------------------------------------------------------------------------------

\definecolor{MyDarkGreen}{rgb}{0.0,0.4,0.0} % This is the color used for comments
\lstloadlanguages{Pascal} % Load Pascal syntax for listings, for a list of other languages supported see: ftp://ftp.tex.ac.uk/tex-archive/macros/latex/contrib/listings/listings.pdf
\lstset{language=Perl, % Use Pascal in this example
        frame=single, % Single frame around code
        basicstyle=\small\ttfamily, % Use small true type font
        keywordstyle=[1]\color{Blue}\bf, % Pascal functions bold and blue
        keywordstyle=[2]\color{Purple}, % Pascal function arguments purple
        keywordstyle=[3]\color{Blue}\underbar, % Custom functions underlined and blue
        identifierstyle=, % Nothing special about identifiers                                         
        commentstyle=\usefont{T1}{pcr}{m}{sl}\color{MyDarkGreen}\small, % Comments small dark green courier font
        stringstyle=\color{Purple}, % Strings are purple
        showstringspaces=false, % Don't put marks in string spaces
        tabsize=5, % 5 spaces per tab
        %
        % Put standard Pascal functions not included in the default language here
        morekeywords={rand},
        %
        % Put Pascal function parameters here
        morekeywords=[2]{on, off, interp},
        %
        % Put user defined functions here
        morekeywords=[3]{test},
        %
        morecomment=[l][\color{Blue}]{...}, % Line continuation (...) like blue comment
        numbers=left, % Line numbers on left
        firstnumber=1, % Line numbers start with line 1
        numberstyle=\tiny\color{Blue}, % Line numbers are blue and small
        stepnumber=5 % Line numbers go in steps of 5
}

% Creates a new command to include a perl script, the first parameter is the filename of the script (without .p), the second parameter is the caption
\newcommand{\pascalscript}[2]{
\begin{itemize}
\item[]\lstinputlisting[caption=#2,label=#1]{#1.p}
\end{itemize}
}

%----------------------------------------------------------------------------------------
%	DOCUMENT STRUCTURE COMMANDS
%	Skip this unless you know what you're doing
%----------------------------------------------------------------------------------------

% Header and footer for when a page split occurs within a problem environment
%\newcommand{\enterProblemHeader}[1]{
%\nobreak\extramarks{#1}{#1 continued on next page\ldots}\nobreak
%\nobreak\extramarks{#1 (continued)}{#1 continued on next page\ldots}\nobreak
%}

% Header and footer for when a page split occurs between problem environments
%\newcommand{\exitProblemHeader}[1]{
%\nobreak\extramarks{#1 (continued)}{#1 continued on next page\ldots}\nobreak
%\nobreak\extramarks{#1}{}\nobreak
%}

\setcounter{secnumdepth}{0} % Removes default section numbers
\newcounter{homeworkProblemCounter} % Creates a counter to keep track of the number of problems

\newcommand{\homeworkProblemName}{}
\newenvironment{homeworkProblem}[1][Aufgabe \arabic{homeworkProblemCounter}]{ % Makes a new environment called homeworkProblem which takes 1 argument (custom name) but the default is "Problem #"
\stepcounter{homeworkProblemCounter} % Increase counter for number of problems
\renewcommand{\homeworkProblemName}{#1} % Assign \homeworkProblemName the name of the problem
\section{\homeworkProblemName} % Make a section in the document with the custom problem count
%\enterProblemHeader{\homeworkProblemName} % Header and footer within the environment
}{
%\exitProblemHeader{\homeworkProblemName} % Header and footer after the environment
}

\newcommand{\problemAnswer}[1]{ % Defines the problem answer command with the content as the only argument
\noindent\framebox[\columnwidth][c]{\begin{minipage}{0.98\columnwidth}#1\end{minipage}} % Makes the box around the problem answer and puts the content inside
}

\newcommand{\homeworkSectionName}{}
\newenvironment{homeworkSection}[1]{ % New environment for sections within homework problems, takes 1 argument - the name of the section
\renewcommand{\homeworkSectionName}{#1} % Assign \homeworkSectionName to the name of the section from the environment argument
\subsection{\homeworkSectionName} % Make a subsection with the custom name of the subsection
%\enterProblemHeader{\homeworkProblemName\ [\homeworkSectionName]} % Header and footer within the environment
}{
%\enterProblemHeader{\homeworkProblemName} % Header and footer after the environment
}

%----------------------------------------------------------------------------------------
%	NAME AND CLASS SECTION
%----------------------------------------------------------------------------------------

\newcommand{\hmwkTitle}{Blatt} % Assignment title
\newcommand{\hmwkDueDate}{25.\ September\ 2015} % Due date
\newcommand{\hmwkClass}{Theoretische Informatik} % Course/class
\newcommand{\hmwkClassInstructor}{} % Teacher/lecturer
\newcommand{\hmwkAuthorName}{Linus Fessler, Markus Hauptner, Philipp Schimmelfennig} % Your name
\newcommand{\hmwkNumber}{1}
\newcommand\eq{\stackrel{\mathclap{\normalfont\mbox{def}}}{=}} 

%----------------------------------------------------------------------------------------
%	TITLE PAGE
%----------------------------------------------------------------------------------------

\title{
\vspace{2in}
\textmd{\textbf{\hmwkClass:\ \hmwkTitle\ \hmwkNumber}}\\
\normalsize\vspace{0.1in}\small{Abgabe\ bis\ \hmwkDueDate}\\
\vspace{0.1in}\large{\textit{\hmwkClassInstructor}
\vspace{3in}
}}
\author{\textbf{\hmwkAuthorName}}
\date{} % Insert date here if you want it to appear below your name

%----------------------------------------------------------------------------------------

\begin{document}

\maketitle

%----------------------------------------------------------------------------------------
%	TABLE OF CONTENTS
%----------------------------------------------------------------------------------------

%\setcounter{tocdepth}{1} % Uncomment this line if you don't want subsections listed in the ToC

\addtocounter{homeworkProblemCounter}{0}
\newpage
%\tableofcontents
%\newpage

%----------------------------------------------------------------------------------------
%	Aufgabe S1
%----------------------------------------------------------------------------------------

\begin{homeworkProblem}

\begin{enumerate}[(a)]
	\item
	Für jede Länge $1$ bis $m$ schauen wir die Anzahl Möglichkeiten an, ein Teilwort zu bilden.\\
	Bei Länge $1$ können wir $m$ Teilwörter bilden, die je bei den Positionen $1$ bis $m$ beginnen.\\
	Bei Länge $2$ können wir $m-1$ Teilwörter bilden, die je bei den Positionen $1$ bis $m-1$ beginnen.\\
	\centerline{\vdots}\\
	Bei Länge $m-1$ können wir zwei Teilwörter bilden, die je bei den Positionen $1$ und $2$ beginnen.\\
	Bei Länge $m$ können wir ein Teilwort bilden, das bei Position $1$ beginnt.\\
	Es gibt also höchstens
	\begin{equation*}
		1+2+\cdots+m = \sum^m_{i=1}i
	\end{equation*}
	verschiedene Teilwörter, falls keine von ihnen gleich sind.
	
	\item
	Fallunterschiedung:
	\begin{itemize}
		\item
		$n=1$: 0 Wörter
		\item
		$n=2$: 0 Wörter
		\item
		$n=3$: $3!$ verschiedene Wörter
		\item
		$n>3$:\\
		Es gibt insgesamt $3^n$ viele verschiedene Wörter.\\
		Es gibt genau 3 Wörter $\{a^n,\, b^n,\, c^n\}$
		die genau einen Buchstaben enthalten.\\
		Es gibt $3\cdot2^n$ viele Wörter, die genau zwei verschiedene Zeichen enthalten.\\
		Die übrigbleibenden $3^n-3-3\cdot 2^n$ Wörter sind die gesuchten, verschiedenen, in denen jeder Buchstabe $\{a,\,b,\,c\}$ einmal vorkommt.
	\end{itemize}
\end{enumerate}
\end{homeworkProblem}

%----------------------------------------------------------------------------------------
%	Aufgabe S2
%----------------------------------------------------------------------------------------
\begin{homeworkProblem}
\begin{enumerate}[(a)]
	
	\item
	Richtig.\\
	Zunächst gilt $(\{a,\,b\}^*)^2=\{a,\,b\}^*$, da jedes Element in $\{a,\,b\}^*$, mit sich selbst konkateniert (was in einer unendlichen Menge möglich ist), in der Menge $(\{a,\,b\}^*)^2$ enthalten ist und für ein $x^2 \in (\{a,\,b\}^*)^2 $ gilt, dass $x \in \{a,\,b\}^*$, also auch wieder $x^2 \in \{a,\,b\}^*$ (weil $\{a,\,b\}^*$ unendlich ist).\\
	$"\supseteq": (\{a\}^*\{b\}^*)^* \eq (\{a^i \mid i\in\mathbb{N}\}\{b^j \mid j\in\mathbb{N}\})^*$. Setzt man einmal $i=1$ und $ j=0$ und einmal $i=0$ und $j=1$ folgt daraus: $(\{a^i \mid i\in\mathbb{N}\}\{b^j \mid j\in\mathbb{N}\})^* \supseteq (\{a\}\{\lambda\}\cup\{\lambda\}\{b\})^* = \{a,\,b\}^*$\\
	$"\subseteq":$
	
	\item
	Falsch. \\
	Zu zeigen: $(\{a\}^*\{b\}^*)^* \neq (\{a,\,b\}^2)^*$\\
	Beweis: Wir zeigen, dass $a$ in $(\{a\}^*\{b\}^*)^*$ ist, aber nicht in $(\{a,\,b\})^2$.
	\begin{alignat}{2}
		&&a&=a\lambda \in \{a\}^*\lambda \subseteq \{a\}^*\{b\}^*\notag\\
		&\Rightarrow \quad
		&a &\in \{a\}^*\{b\}^*\notag
	\end{alignat}
	Beweis für $\lambda \in \{a\}^*\{b\}^*$ analog.
	\begin{alignat}{2}
		&\Rightarrow\quad
		&a&=a\lambda \in \{\{a\}^*\{b\}^*\}\{\{a\}^*\{b\}^*\} = \{\{a\}^*\{b\}^*\}^2 \subseteq \{\{a\}^*\{b\}^*\}^*\notag\\
		&\Rightarrow\quad
		&a &\in \{\{a\}^*\{b\}^*\}^*\notag
	\end{alignat}
	Zu zeigen: $a\not\in (\{a,\,b\})^2 = \{aa,\,ab,\,bb,\,ba\}^*=L^*$\\
	Begründung: Das Wort $a$ hat Länge 1. Jedes Element in $L$ hat Länge 2. Durch Konkatenation mit beliebiger Potenz liegen in $L^*$ Wörter mit Länge $>2$ und $\lambda$ mit Länge $0$. Aber kein Wort mit Länge 1.
	 
	\item Richtig.
		\begin{align*}
			L_2\cdot (L_2-L_1) &= \{xy | x\in L_2 \wedge y\in L_2-L_1\}&\text{(Def. Konkatenation)}\\
			&= \{xy | x\in L_2 \wedge y\in L_2 \wedge y \not\in L_1\}&(\text{Def. Subtraktion})\\
			&= \{xy | (x\in L_2 \wedge y\in L_2) \wedge (x\in L_2 \wedge y \not\in L_1)\}&(A=A\land A)\\
			&=\{xy|x\in L_2 \wedge y \in L_2\} - \{xy|x\in L_2 \wedge y \in L_1\}&(\text{Def. Subtraktion})\\
			&=(L_2)^2 - L_2\cdot L_1&(\text{Def. Konkatenation})\\
		\end{align*} 
	
\end{enumerate}
\end{homeworkProblem}
\begin{homeworkProblem} 
 

\begin{enumerate}[(a)]
	\item 
	Behauptung: $L=\{ab\}^*$ 
	\begin{itemize} 


		\item 
		Zu zeigen: L ist eine Sprache: $L \subseteq \Sigma ^*$ \\ 
		Beweis: 
		\begin{align*}
 		\Sigma = \{a,b\} \Rightarrow \Sigma ^* = \{a,b\}^* \eq \bigcup_{i\in\mathbb{N}}\{a,b\}^i = \bigcup_{\substack{i\in\mathbb{N}\\\text{i gerade}}} \{a,b\}^i \cup \bigcup_{\substack{i\in\mathbb{N}\\\text{i ungerade}}} \{a,b\}^i \Rightarrow \bigcup_{\substack{i\in\mathbb{N}\\\text{i gerade}}} \{a,b\}^i \subseteq \Sigma^*
		\end{align*}


		\item 
		\begin{itemize} 
			Zu zeigen bleibt: $L \subseteq \bigcup_{i\in\mathbb{N}\text{, i gerade}}\{a,b\}^i$ \\ 
			Beweis: 
			\begin{align*}
 			\bigcup_{i\in\mathbb{N}\text{, i gerade}}\{a,b\}^i = \bigcup_{k\in\mathbb{N}} \{a,b\}^{2k} = \bigcup_{k\in\mathbb{N}} (\{a,b\}^2)^{k} = \bigcup_{k\in\mathbb{N}} \{aa,ab,ba,bb\}^{k} \\ 
 			\bigcup_{k\in\mathbb{N}} \{ab\}^{k} \subseteq \bigcup_{k\in\mathbb{N}} \{aa,ab,ba,bb\}^{k} \Rightarrow
 			\bigcup_{k\in\mathbb{N}} \{ab\}^{k} \subseteq
 			\bigcup_{i\in\mathbb{N}\text{, i gerade}}\{a,b\}^i
			\end{align*}
		\end{itemize} 
		$\Rightarrow L \subseteq \Sigma^*$

		\item
		Zu zeigen: $L^i=L \quad\forall i \in \mathbb{N}-\{0,\,1\}$\\
		Für alle $i \geq 1$
		\begin{align}
			L^i &= \{x\,|\,x=v_1\cdot v_2\cdots v_i \wedge v_j \in L \;\forall j\}\\
			\Rightarrow L^i &=\{x\,|\,x=(ab)^{k_1}\cdot (ab)^{k_2}\cdots (ab)^{k_i}\}\\
			\Rightarrow L^i &=\{x\,|\,x=(ab)^{(k_1+k_2+\cdots+k_i)}=(ab)^{k'}\} = \{ab\}^* = L			
		\end{align} 
		gilt, da L nur ein Zeichen $ab$ enthält, $k'=k_1+k_2+\cdots+k_i$ und $k'$ beliebig in $\mathbb{N}$ sein kann.
		\item
		Zu zeigen: $L \neq \{\lambda\}^*$ \\ 
		Beweis: $\{\lambda\}^* \eq \{\lambda^i \mid i\in\mathbb{N}\} = \{\lambda, \lambda^2, \lambda^3, \cdots\} = \{\lambda, \lambda, \lambda, \cdots\} = \{\lambda\}$ \\ 
		$L = \{ab\}^* \eq \{(ab)^i \mid i\in\mathbb{N}\} = \{\lambda, ab, abab, ababab, \cdots\}$ \\ 
		Damit ist $L \neq \{\lambda\}^*$ (da z.B. $ab \in L$, aber $ab \notin \{\lambda\})$ 


		\item 
		Zu zeigen: $L \neq \{a\}^*$ \\ 
		Beweis: $\{a\}^* \eq \{a^i \mid i\in\mathbb{N}\}$, enthält insbesondere keine Wörter, die den Buchstaben $b$ enthalten. \\ 
		In $L = \{ab\}^* \eq \{(ab)^i \mid i\in\mathbb{N}\}$ gibt es allerdings Wörter, die $b$ enthalten, womit $L = \{ab\}^* \neq \{a\}^*$ gelten muss. 


		\item 
		Zu zeigen: $L \neq \{b\}^*$ \\ 
		Beweis: Analog zu $L \neq \{a\}^*$. 


		\item 
		Zu zeigen: $L \neq \{a,b\}^*$ \\ 
		Beweis: $\{a,b\}^* \eq \{\lambda, a, b, aa, ba, ab, bb, \cdots\}$, insbesondere gilt $a,b \in \{a,b\}^*$. \\ 
		Dagegen ist $L = \{ab\}^* \eq \{(ab)^i \mid i\in\mathbb{N}\} = \{\lambda, ab, abab, ababab, \cdots\}$, womit gilt $a,b \notin \{ab\}^*$. \\ 
		Daraus folgt, dass $L \neq \{a,b\}^*$ gelten muss. 


	\end{itemize} 


	\item[(b)] 
	Behauptung: Es gibt keine nichtleere endliche Sprache $L \neq {\lambda}$ über dem Alphabet $\{a,b\}$, die die Bedingung $L^2 = L$ erfüllt. \\ 
	Beweis: Sei $L$ eine nichtleere endliche Sprache $L \neq {\lambda}$. Dann $\exists l \in L: l=\max L$. Das Wort $ll$ muss in $L^2$ enthalten sein (und ist sogar das längste Wort in $L^2$). Sei $|l| = k \Rightarrow |ll| = 2k$. Da $k \neq 0$, ist das längste Wort in $L^2$ doppelt so lang wie das längste Wort in $L$, womit $L^2 \neq L$ ist. 
\end{enumerate} 


\end{homeworkProblem} 


\end{document}
