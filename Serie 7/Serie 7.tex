%BEGIN_FOLD
%%%%%%%%%%%%%%%%%%%%%%%%%%%%%%%%%%%%%%%%
% Programming/Coding Assignment
% LaTeX Template
%
% Original author:
% Ted Pavlic (http://www.tedpavlic.com)
%
%
% This template uses a Perl script as an example snippet of code, most other
% languages are also usable. Configure them in the "CODE INCLUSION 
% CONFIGURATION" section.
%
%%%%%%%%%%%%%%%%%%%%%%%%%%%%%%%%%%%%%%%%%
%END_FOLD
%----------------------------------------------------------------------------------------
%BEGIN_FOLD	%PACKAGES AND OTHER DOCUMENT CONFIGURATIONS
%----------------------------------------------------------------------------------------

\documentclass{article}
\usepackage[utf8]{inputenc}

\usepackage[ngerman]{babel}
\usepackage{amsmath}
\usepackage{amssymb}
\usepackage{mathtools}
\usepackage{amsfonts}
\usepackage{fancyhdr} % Required for custom headers
\usepackage{lastpage} % Required to determine the last page for the footer
\usepackage{extramarks} % Required for headers and footers
\usepackage[usenames,dvipsnames]{color} % Required for custom colors
\usepackage{graphicx} % Required to insert images
\usepackage{listings} % Required for insertion of code
\usepackage{courier} % Required for the courier font
\usepackage{enumerate} % used for enumerate args
\usepackage{multicol} % columns

\usepackage{pgf} 
\usepackage{tikz}
\usepackage{forest} % treees :D
\usetikzlibrary{arrows,automata} %for FSM
\usepackage{mathtools}
\usepackage{wasysym}

% Custom commands
\DeclareMathOperator{\Space}{Space}
\DeclareMathOperator{\Time}{Time}
\DeclareMathOperator{\Kl}{Kl} %Klassen von Zuständen
\newcommand{\Ebool}{\{0, 1\}}
\newcommand{\irow}[1]{% inline row vector
	\begin{smallmatrix}(#1)\end{smallmatrix}
}
\newcommand{\icol}[1]{% inline column vector
	\left(\begin{smallmatrix}#1\end{smallmatrix}\right)%
}
\DeclarePairedDelimiter{\ceil}{\lceil}{\rceil}
% Shamelessly copied from http://tex.stackexchange.com/questions/43008/absolute-value-symbols
\DeclarePairedDelimiter\abs{\lvert}{\rvert} % nice |x|
\DeclarePairedDelimiter\norm{\lVert}{\rVert} % nice ||x||
% Swap the definition of \abs* and \norm*, so that \abs
% and \norm resizes the size of the brackets, and the 
% starred version does not.
%END_FOLD
%----------------------------------------------------------------------------------------
%BEGIN_FOLD
\makeatletter
\let\oldabs\abs
\def\abs{\@ifstar{\oldabs}{\oldabs*}}
\let\oldnorm\norm
\def\norm{\@ifstar{\oldnorm}{\oldnorm*}}
\makeatother


% Margins
\topmargin=-0.45in
\evensidemargin=0in
\oddsidemargin=0in
\textwidth=6.5in
\textheight=9.0in
\headsep=0.25in

\linespread{1.1} % Line spacing

% Set up the header and footer
\pagestyle{fancy}
\lhead{\hmwkAuthorName} % Top left header
%\chead{\hmwkClass\ (\hmwkClassInstructor\): \hmwkTitle} % Top center head
%\rhead{\firstxmark} % Top right header
\rhead{}
\lfoot{\lastxmark} % Bottom left footer
\cfoot{} % Bottom center footer
\rfoot{Seite\ \thepage\ von\ \protect\pageref{LastPage}} % Bottom right footer
\renewcommand\headrulewidth{0.4pt} % Size of the header rule
\renewcommand\footrulewidth{0.4pt} % Size of the footer rule

\setlength\parindent{0pt} % Removes all indentation from paragraphs

%----------------------------------------------------------------------------------------
%	CODE INCLUSION CONFIGURATION
%----------------------------------------------------------------------------------------

\definecolor{MyDarkGreen}{rgb}{0.0,0.4,0.0} % This is the color used for comments
\lstloadlanguages{Pascal} % Load Pascal syntax for listings, for a list of other languages supported see: ftp://ftp.tex.ac.uk/tex-archive/macros/latex/contrib/listings/listings.pdf
\lstset{language=Perl, % Use Pascal in this example
        frame=single, % Single frame around code
        basicstyle=\small\ttfamily, % Use small true type font
        keywordstyle=[1]\color{Blue}\bf, % Pascal functions bold and blue
        keywordstyle=[2]\color{Purple}, % Pascal function arguments purple
        keywordstyle=[3]\color{Blue}\underbar, % Custom functions underlined and blue
        identifierstyle=, % Nothing special about identifiers                                         
        commentstyle=\usefont{T1}{pcr}{m}{sl}\color{MyDarkGreen}\small, % Comments small dark green courier font
        stringstyle=\color{Purple}, % Strings are purple
        showstringspaces=false, % Don't put marks in string spaces
        tabsize=5, % 5 spaces per tab
        %
        % Put standard Pascal functions not included in the default language here
        morekeywords={rand},
        %
        % Put Pascal function parameters here
        morekeywords=[2]{on, off, interp},
        %
        % Put user defined functions here
        morekeywords=[3]{test},
        %
        morecomment=[l][\color{Blue}]{...}, % Line continuation (...) like blue comment
        numbers=left, % Line numbers on left
        firstnumber=1, % Line numbers start with line 1
        numberstyle=\tiny\color{Blue}, % Line numbers are blue and small
        stepnumber=5 % Line numbers go in steps of 5
}

% Creates a new command to include a perl script, the first parameter is the filename of the script (without .p), the second parameter is the caption
\newcommand{\pascalscript}[2]{
\begin{itemize}
\item[]\lstinputlisting[caption=#2,label=#1]{#1.p}
\end{itemize}
}

%----------------------------------------------------------------------------------------
%	DOCUMENT STRUCTURE COMMANDS
%	Skip this unless you know what you're doing
%----------------------------------------------------------------------------------------

% Header and footer for when a page split occurs within a problem environment
%\newcommand{\enterProblemHeader}[1]{
%\nobreak\extramarks{#1}{#1 continued on next page\ldots}\nobreak
%\nobreak\extramarks{#1 (continued)}{#1 continued on next page\ldots}\nobreak
%}

% Header and footer for when a page split occurs between problem environments
%\newcommand{\exitProblemHeader}[1]{
%\nobreak\extramarks{#1 (continued)}{#1 continued on next page\ldots}\nobreak
%\nobreak\extramarks{#1}{}\nobreak
%}

\setcounter{secnumdepth}{0} % Removes default section numbers
\newcounter{homeworkProblemCounter} % Creates a counter to keep track of the number of problems

\newcommand{\homeworkProblemName}{}
%END_FOLD
\newenvironment{homeworkProblem}[1][Aufgabe \arabic{homeworkProblemCounter}]
%BEGIN_FOLD
{ % Makes a new environment called homeworkProblem which takes 1 argument (custom name) but the default is "Problem #"

\stepcounter{homeworkProblemCounter} % Increase counter for number of problems
\renewcommand{\homeworkProblemName}{#1} % Assign \homeworkProblemName the name of the problem
\section{\homeworkProblemName} % Make a section in the document with the custom problem count
%\enterProblemHeader{\homeworkProblemName} % Header and footer within the environment
}{
%\exitProblemHeader{\homeworkProblemName} % Header and footer after the environment
}

\newcommand{\problemAnswer}[1]{ % Defines the problem answer command with the content as the only argument
\noindent\framebox[\columnwidth][c]{\begin{minipage}{0.98\columnwidth}#1\end{minipage}} % Makes the box around the problem answer and puts the content inside
}

\newcommand{\homeworkSectionName}{}
\newenvironment{homeworkSection}[1]{ % New environment for sections within homework problems, takes 1 argument - the name of the section
\renewcommand{\homeworkSectionName}{#1} % Assign \homeworkSectionName to the name of the section from the environment argument
\subsection{\homeworkSectionName} % Make a subsection with the custom name of the subsection
%\enterProblemHeader{\homeworkProblemName\ [\homeworkSectionName]} % Header and footer within the environment
}{
%\enterProblemHeader{\homeworkProblemName} % Header and footer after the environment
}
%END_FOLD
%----------------------------------------------------------------------------------------
%BEGIN_FOLD %NAME AND CLASS SECTION
%----------------------------------------------------------------------------------------

\newcommand{\hmwkTitle}{Blatt} % Assignment title
\newcommand{\hmwkDueDate}{9.\ Oktober\ 2015} % Due date
\newcommand{\hmwkClass}{Theoretische Informatik} % Course/class
\newcommand{\hmwkClassInstructor}{} % Teacher/lecturer
\newcommand{\hmwkAuthorName}{Linus Fessler, Markus Hauptner, Philipp Schimmelfennig} % Your name
\newcommand{\hmwkNumber}{7}
\newcommand{\hmwkAssiInfo}{Assistent: Sacha Krug, CHN D 42}
%END_FOLD
%----------------------------------------------------------------------------------------
%BEGIN_FOLD
%----------------------------------------------------------------------------------------
%	TITLE PAGE
%----------------------------------------------------------------------------------------

\title{
\vspace{2in}
\textmd{\textbf{\hmwkClass:\ \hmwkTitle\ \hmwkNumber}}\\
\normalsize\vspace{0.1in}\small{Abgabe\ bis\ \hmwkDueDate}
\\\hmwkAssiInfo
\\
\vspace{0.1in}\large{\textit{\hmwkClassInstructor}
\vspace{3in}
}}
\author{\textbf{\hmwkAuthorName}}
\date{} % Insert date here if you want it to appear below your name

%----------------------------------------------------------------------------------------
%END_FOLD
\begin{document}
%BEGIN_FOLD
\maketitle

%----------------------------------------------------------------------------------------
%	TABLE OF CONTENTS
%----------------------------------------------------------------------------------------

%\setcounter{tocdepth}{1} % Uncomment this line if you don't want subsections listed in the ToC
%END_FOLD
%----------------------------------------------------------------------------------------
\addtocounter{homeworkProblemCounter}{18}
%----------------------------------------------------------------------------------------
\newpage
%\tableofcontents
%\newpage

%----------------------------------------------------------------------------------------
%	Aufgabe S1
%----------------------------------------------------------------------------------------

\begin{homeworkProblem}
	
\end{homeworkProblem}
\begin{homeworkProblem}
	\begin{enumerate}[(a)]
		\item
		$e(n)=2^n$\\
		Wir konstruieren eine 2-Band Turingmaschine $M$. $M$ bekommt als Eingabe das Wort $0^n$ auf Band 0. 
		Zu Beginn schreibt $M$ eine $0$ auf \textit{Band 1}.
		Solange der Lesekopf des Eingabebandes nicht \$ liest:
		\begin{enumerate}[1.]
			\item
			Gehe auf \textit{Band 1} nach links bis $\cent$. 
			\item
			Gehe auf \textit{Band 2} nach links bis $\cent$
			\item
			Lies Zeichen auf \textit{Band 1}. Schreibe für jede gelesene $0$ auf \textit{Band 1} $00$ auf \textit{Band 2}. Für ein $\textvisiblespace$ schreibe ein $\textvisiblespace$.
			\item
			Gehe auf Beiden Bändern nach links und kopiere Inhalt von \textit{Band 2} auf \textit{Band 1} einschließlich bis Zeichen $\textvisiblespace$.
			\item
			Rücke mit Lesekopf nach rechts.
		\end{enumerate}
		Das Ergebnis steht dann auf \textit{Band 2} bis zum ersten $\textvisiblespace$.
		
		Auf diese Art generieren wir $2^n$ $0$en.
		Für n $0$en der Eingabe lesen wir pro Schritt $2^i$ Nullen. Das schreiben geschieht jeweils in $\mathcal{O}(1)$.
		\[
			\sum_{i=1}^{n}2^i = 2^{n+1} - 2 \in \mathcal{O}(2^n) 
		\]
		Folglich ist $e(n)$ zeitkonstruierbar.\\
		
		\item 
		$f(n)=fib_n$\\
		Wir konstruieren ein 3-Band Turingmaschine $M$. $M$ bekommt als Eingabe das Wort $0^n$ auf Band 0. 
		Wir unterscheiden mehrere Eingaben $w$.
		\begin{enumerate}[{Fall} 1]
			\item $w=\lambda$\\
			In diesem Fall ist $n=0$. $M$ schreibt $0$ auf Band 1 und hält.
			\item $w=0$\\
			In diesem Fall ist $n=1$. $M$ schreibt $1$ auf Band 1 und hält.
			\item $|w|=n, n \geq 2$
			Der Lesekopf auf Band $0$ liegt auf der dritten $0$.
			\begin{enumerate}[1.]
				\item
				$M$ schreibt $\lambda$ auf \textit{Band 1} und $0$ auf \textit{Band 2}.
				\item
				$M$ löscht \textit{Band 3} und schreibt zuerst alle $0$en von \textit{Band 1} und dann alle $0$en von \textit{Band 2} auf \textit{Band 3}.
				\item
				Der Lesekopf für \textit{Band 0} geht nach rechts. Liest er dort \$ ist auf \textit{Band 3} das Ergebnis und $M$ hält.
				Ansonsten kopiert $M$ den Inhalt von \textit{Band 2} auf \textit{Band 1} und den von \textit{Band 3} auf \textit{Band 2}. Dann wird zu Schritt $2.$ gesprungen.
			\end{enumerate}
		\end{enumerate}
		ANALyse fehlt noch.
	\end{enumerate}
\end{homeworkProblem}
\begin{homeworkProblem}
	Wir wissen: $\; f:\mathbb{N}\rightarrow\mathbb{N},\;\; g:\mathbb{N}\rightarrow\mathbb{N}\;\;$ und $f$ und $g$ sind beide platzkonstruierbar.\\
	$\Rightarrow$ Es gibt 1-Band-Turingmaschinen $F$ unf $G$, so dass  $\begin{array}{ll}
	\Space_F(n_1)\leq s_1(n_1)\\
	\Space_G(n_2)\leq s_2(n_2)\\
	\end{array} \quad\forall n_1, n_2 \in \mathbb{N}$\\
	und für jede Eingabe 
	$\begin{array}{ll}
	0^{n_1} \text{ generiert $F$ das Wort } 0^{s_1(n_1)}\\
	0^{n_2} \text{ generiert $G$ das Wort } 0^{s_2(n_2)}
	\end{array}$
	auf ihrem Arbeitsband und hält in Zustand $q_{accept}$.\\
	Sei nun $H$ eine 1-Band-Turingmaschine mit Eingabe mit Arbeitsalphabet $\Gamma_F\times\Gamma_G \cup \Gamma_F \cup \Gamma_G$ und Eingabe $0^n$.\\
	$H$ simuliert nun die Arbeit von $F$ auf folgende Weise:\\
	Ist der Lesekopf des Arbeitsbandes auf dem Symbol $\icol{\alpha\\\beta}$ simuliert $M$ $F$ so als würde $F$ $\alpha$ lesen. Schreibt $F$ ein neues Zeichen $\alpha'$, schreibt $H$ $\icol{\alpha'\\\beta}$ auf das Band.\\
	Sobald $F$ gehalten hat, fährt $H$ mit Eingabe- und Arbeitsbandkopf nach links auf $\cent$.\\
	Anschließend simuliert $H$, $G$ auf gleiche Art wie $F$.\\
	Für $\icol{\alpha\\\beta}$ wird $\beta$ gelesen, und für $\beta'$ wird an der gleichen Stelle $\icol{\alpha\\\beta'}$ geschrieben.\\
	$H$ wandelt nun die das Band von der Form 
	\[
		\cent\icol{0\\0}\icol{0\\0}\cdots\icol{\textvisiblespace\\0}
	\]
	Schrittweise um in $0^l$, wie folgt:
	\begin{enumerate}[1.]
		\item
		Ist Zeichen der Form $\icol{0\\0}$, ersetze durch $0$ und ersetze erstes $\textvisiblespace$ durch eine $0$.
		\item Ist Zeichen der Form $\icol{\textvisiblespace\\0)}$ oder $\icol{0\\\textvisiblespace}$ ersetze mit $0$.
	\end{enumerate}
	Wiederhole Schritte bis von links zeichen $\textvisiblespace$ gelesen wird. Dann akzeptiere.\\
	Es steht nun genau $s_1(n)+s_2(n) := s_3(n)$ auf dem Band und der benutzte Platze $\Space_H = \max{s_1(n), s_2(n)} \leq s_1(n)+s_2(n) = s_3(n)$.\\
	$\Rightarrow$ $H$ ist platzkonstruierbar.
\end{homeworkProblem}

\end{document}
